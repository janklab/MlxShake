% This LaTeX was auto-generated from MATLAB code.
% To make changes, update the MATLAB code and export to LaTeX again.

\documentclass{article}

\usepackage[utf8]{inputenc}
\usepackage[T1]{fontenc}
\usepackage{lmodern}
\usepackage{graphicx}
\usepackage{color}
\usepackage{hyperref}
\usepackage{amsmath}
\usepackage{amsfonts}
\usepackage{epstopdf}
\usepackage[table]{xcolor}
\usepackage{matlab}

\sloppy
\epstopdfsetup{outdir=./}
\graphicspath{ {./UserGuide_images/} }

\begin{document}

\matlabtitle{ExportMlx User Guide}

\matlabheading{Introduction}

\begin{par}
\begin{flushleft}
HELLO WORLD!
\end{flushleft}
\end{par}

\begin{par}
\begin{flushleft}
This project provides a function \texttt{latex2markdown} that supports converting Matlab live scripts to Markdown files. This is useful in generating rich documentation for Matlab programs.
\end{flushleft}
\end{par}

\begin{par}
\begin{flushleft}
This README is itself generated from a live script using ExportMlx!
\end{flushleft}
\end{par}

\matlabheading{Installation and Setup}

\begin{par}
\begin{flushleft}
Download the distro or repo and put it somewhere on your disk.
\end{flushleft}
\end{par}

\begin{par}
\begin{flushleft}
Add its \texttt{Mcode/} directory to your Matlab path using \texttt{addpath()}.
\end{flushleft}
\end{par}

\matlabheading{Usage}

\matlabheadingthree{Step 1: Manually export your live script to LaTeX using Matlab}

\begin{par}
\begin{flushleft}
\includegraphics[width=\maxwidth{29.001505268439537em}]{image_0}
\end{flushleft}
\end{par}

\begin{par}
\begin{flushleft}
When \texttt{README.tex} is generated by clicking "Export to LaTeX", all the related images, such as included images and generated figures, are saved under the folder \texttt{README\_images} next to the \texttt{.tex} file.
\end{flushleft}
\end{par}

\begin{par}
\begin{flushleft}
\textbf{WARNING}: When exporting to LaTeX right after running the live script, it's observed that the figures will be exported as EPS files or not at all if the live script contains more than 20 figures. I suggest that \textbf{you close the script, reopen it, and then export to LaTeX.}
\end{flushleft}
\end{par}

\begin{par}
\begin{flushleft}
This part also can be done programmatically with a function call (see this \href{https://jp.mathworks.com/matlabcentral/answers/396348-how-to-find-and-replace-within-mlx-live-scripts-across-multiple-files}{MATLAB Answer}), but it's a undocumented internal Matlab function. (This will hopefully soon be supported programmatically by ExportMlx!)
\end{flushleft}
\end{par}

\matlabheadingthree{Step 2: Convert the LaTeX to Markdown using ExportMlx}

\begin{verbatim}
latex2markdown('README');
\end{verbatim}
\begin{par}
\begin{flushleft}
This will generate \texttt{README.md}, a Markdown file suitable for GitHub.
\end{flushleft}
\end{par}

\begin{par}
\begin{flushleft}
The \texttt{latex2markdown} function supports the following options:
\end{flushleft}
\end{par}

\begin{itemize}
\setlength{\itemsep}{-1ex}
   \item{\begin{flushleft} \texttt{'format'}: Style of Markdown to generate. May be \texttt{'github'} (default) or \texttt{'qiita'}  \end{flushleft}}
   \item{\begin{flushleft} \texttt{'outputfilename'}: The name of the Markdown file to be generated. If unspecified, will be the same as the live script, but with an \texttt{.md} file extension. \end{flushleft}}
   \item{\begin{flushleft} \texttt{'png2jpeg'}: Convert PNG images to JPEG images to save space at the expense of image quality. May be \texttt{false} (default) or \texttt{true}. \end{flushleft}}
\end{itemize}

\begin{par}
\begin{flushleft}
Example: A Markdown file for Qiita, named \texttt{QiitaDraft.md}, will be generated by the following command:
\end{flushleft}
\end{par}

\begin{verbatim}
latex2markdown('README', 'format', 'qiita', 'outputfilename', 'QiitaDraft');
\end{verbatim}
\begin{par}
\begin{flushleft}
Note: Qiita is a tech blog platform in Japanese. Qiita uses Markdown for its posts.
\end{flushleft}
\end{par}

\matlabheadingtwo{Differences Between Qiita and GitHub Formats}

\begin{par}
\begin{flushleft}
One is the equations and the other is how to insert the image files. 
\end{flushleft}
\end{par}

\begin{par}
\begin{flushleft}
Qiita allows you to use LaTeX to represent equations (like GitLab?) whereas GitHub does not. For GitHub, ExportMlx uses CODECOGS (\href{https://latex.codecogs.com}{https://latex.codecogs.com}) to render the equations as images.
\end{flushleft}
\end{par}

\begin{par}
\begin{flushleft}
For images, you can just push the image folders and then the README reads them, but you need to drag \& drop your images inside the Qiita Editor.
\end{flushleft}
\end{par}

\matlabheading{Supported Syntax in Live Scripts}

\label{H_08F222FE}
\matlabheadingtwo{MATLAB Code}

\begin{par}
\begin{flushleft}
MATLAB code and its output and figures will be shown as follows:
\end{flushleft}
\end{par}

\begin{matlabcode}
% Matlab code
x = linspace(0,2*pi,100);
y = sin(x)
\end{matlabcode}
\begin{matlaboutput}
y = 1x100    
         0    0.0634    0.1266    0.1893    0.2511    0.3120    0.3717    0.4298    0.4862    0.5406    0.5929    0.6428    0.6901    0.7346    0.7761    0.8146    0.8497    0.8815    0.9096    0.9341    0.9549    0.9718    0.9848    0.9938    0.9989    0.9999    0.9969    0.9898    0.9788    0.9638    0.9450    0.9224    0.8960    0.8660    0.8326    0.7958    0.7557    0.7127    0.6668    0.6182    0.5671    0.5137    0.4582    0.4009    0.3420    0.2817    0.2203    0.1580    0.0951    0.0317

\end{matlaboutput}
\begin{matlabcode}
% Figures
plot(x,y);
\end{matlabcode}
\begin{center}
\includegraphics[width=\maxwidth{56.196688409433015em}]{figure_0.png}
\end{center}

\matlabheadingtwo{Table display}

\begin{par}
\begin{flushleft}
The display of \texttt{table} arrays will be formated like so:
\end{flushleft}
\end{par}

\begin{matlabcode}
array2table(rand(3,4))
\end{matlabcode}
\begin{matlabtableoutput}
{
\begin{tabular} {|c|c|c|c|c|}\hline
\mlcell{ } & \mlcell{Var1} & \mlcell{Var2} & \mlcell{Var3} & \mlcell{Var4} \\ \hline
\mlcell{1} & \mlcell{0.6991} & \mlcell{0.5472} & \mlcell{0.2575} & \mlcell{0.8143} \\ \hline
\mlcell{2} & \mlcell{0.8909} & \mlcell{0.1386} & \mlcell{0.8407} & \mlcell{0.2435} \\ \hline
\mlcell{3} & \mlcell{0.9593} & \mlcell{0.1493} & \mlcell{0.2543} & \mlcell{0.9293} \\ 
\hline
\end{tabular}
}
\end{matlabtableoutput}

\begin{par}
\begin{flushleft}
BUG: If the table contains multicolumn variables, the format is not perfect. Column headings will not be placed correctly.
\end{flushleft}
\end{par}

\begin{matlabcode}
table(rand(3,4))
\end{matlabcode}
\begin{matlabtableoutput}
{
\begin{tabular} {|c|c|c|c|c|}\hline
\mlcell{ } & \multicolumn{4}{|c|}{\mlcell{Var1}} \\ \hline
\mlcell{1} & \mlcell{0.3500} & \mlcell{0.6160} & \mlcell{0.8308} & \mlcell{0.9172} \\ \hline
\mlcell{2} & \mlcell{0.1966} & \mlcell{0.4733} & \mlcell{0.5853} & \mlcell{0.2858} \\ \hline
\mlcell{3} & \mlcell{0.2511} & \mlcell{0.3517} & \mlcell{0.5497} & \mlcell{0.7572} \\ 
\hline
\end{tabular}
}
\end{matlabtableoutput}

\begin{par}
\begin{flushleft}
(Any suggestions to handle merged cells in Markdown are appreciated!)
\end{flushleft}
\end{par}

\label{H_E1D8EAFF}
\matlabheadingtwo{Code Examples}

\begin{par}
\begin{flushleft}
"MATLAB" Code Examples (as opposed to actual executable Matlab code) are rendered as follows:
\end{flushleft}
\end{par}

\begin{verbatim}
% Matlab Code Example display
x = linspace(0,1,100);
y = sin(x);
plot(x,y);
\end{verbatim}
\begin{par}
\begin{flushleft}
"Plain" Code Examples look like this:
\end{flushleft}
\end{par}

\begin{verbatim}
# Python code
print('Hello, world!')
\end{verbatim}
\matlabheadingtwo{Inline Images}

\begin{par}
\begin{flushleft}
Here's a display of an inline image (an image pasted into the live script, as opposed to a figure display).
\end{flushleft}
\end{par}

\begin{par}
\begin{flushleft}
\includegraphics[width=\maxwidth{32.212744606121426em}]{image_1}
\end{flushleft}
\end{par}

\matlabheadingtwo{Equations}

\begin{par}
\begin{flushleft}
Any equations in live scripts will be exported as LaTeX. For GitHub, CODECOGS helps display them by rendering them as images.
\end{flushleft}
\end{par}


\vspace{1em}
\begin{par}
\begin{flushleft}
Here's an inline equation: $\sin^2 x+\cos^2 x=1$. 
\end{flushleft}
\end{par}

\begin{par}
\begin{flushleft}
If you have multiple lines of equations:
\end{flushleft}
\end{par}

\begin{par}
$$\begin{array}{l}
\sin x=-\int \cos xdx\\
\cos x=\int \sin xdx
\end{array}$$
\end{par}

\begin{par}
\begin{flushleft}
Single line equations look like this:
\end{flushleft}
\end{par}

\begin{par}
$$\sin x=-\int \cos xdx$$
\end{par}

\matlabheadingtwo{Lists}

\begin{par}
\begin{flushleft}
Here's a unordered list:
\end{flushleft}
\end{par}

\begin{itemize}
\setlength{\itemsep}{-1ex}
   \item{\begin{flushleft} Item foo \end{flushleft}}
   \item{\begin{flushleft} Item bar \end{flushleft}}
   \item{\begin{flushleft} Item baz \end{flushleft}}
\end{itemize}

\begin{par}
\begin{flushleft}
Here's an ordered list:
\end{flushleft}
\end{par}

\begin{enumerate}
\setlength{\itemsep}{-1ex}
   \item{\begin{flushleft} Item one \end{flushleft}}
   \item{\begin{flushleft} Item two \end{flushleft}}
   \item{\begin{flushleft} Item three \end{flushleft}}
\end{enumerate}

\matlabheadingtwo{Font Styles}

\begin{par}
\begin{flushleft}
Here are the complete list of font styles available in Live Editor:
\end{flushleft}
\end{par}

\begin{itemize}
\setlength{\itemsep}{-1ex}
   \item{\begin{flushleft} \textbf{Bold} \end{flushleft}}
   \item{\begin{flushleft} \textit{Italic} \end{flushleft}}
   \item{\begin{flushleft} \underline{Underline} \end{flushleft}}
   \item{\begin{flushleft} \texttt{Monospace}. \end{flushleft}}
\end{itemize}

\begin{par}
\begin{flushleft}
And some conbinations of styles:
\end{flushleft}
\end{par}

\begin{itemize}
\setlength{\itemsep}{-1ex}
   \item{\begin{flushleft} \textit{\textbf{BoldItalic}} \end{flushleft}}
   \item{\begin{flushleft} \underline{\textbf{BoldUnderline}} \end{flushleft}}
   \item{\begin{flushleft} \texttt{\textbf{BoldMonospace}} \end{flushleft}}
   \item{\begin{flushleft} \underline{\textit{ItalicUnderline}} \end{flushleft}}
   \item{\begin{flushleft} \texttt{\textit{ItalicMonospace}} \end{flushleft}}
   \item{\begin{flushleft} \texttt{\underline{UnderlineMonospace}} \end{flushleft}}
   \item{\begin{flushleft} \underline{\textit{\textbf{BoldItalicUnderline}}} \end{flushleft}}
   \item{\begin{flushleft} \texttt{\textit{\textbf{BoldItalicMonospace}}} \end{flushleft}}
   \item{\begin{flushleft} \texttt{\underline{\textbf{BoldUnderlineMonospace}}} \end{flushleft}}
   \item{\begin{flushleft} \texttt{\underline{\textit{ItalicUnderlineMonospace}}} \end{flushleft}}
\end{itemize}

\begin{par}
\begin{flushleft}
Note that underlines do not show up in the Markdown! (I think this is because Markdown doesn't have markup for underlines?)
\end{flushleft}
\end{par}

\matlabheadingtwo{Quotations}

\begin{par}
\begin{flushleft}
There's not corresponding function, but here centered paragraph is treated as quotation.
\end{flushleft}
\end{par}

\begin{par}
\begin{center}
There's not corresponding function, but here centered paragraph is treated as quotation.
\end{center}
\end{par}

\matlabheading{Feedback and Support}

\begin{par}
\begin{flushleft}
Hope this accelerates your Matlab life! Any comments and suggestions are welcome. Visit the project repo at \href{https://github.com/janklab/ExportMlx}{https://github.com/janklab/ExportMlx}.
\end{flushleft}
\end{par}

\end{document}
